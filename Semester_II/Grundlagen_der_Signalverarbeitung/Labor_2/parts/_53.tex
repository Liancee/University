\begin{enumerate}[label=\alph*)]
	\item Darstellung eines diskreten Impulskamms mit $n=10$ $\delta$-Impulsen in dem Intervall $L=10$ \\
	\addimg{./assets/Lab2_53a.png}
	
	\item Darstellung einer diskreten Sinus-Schwingung $s(t_n) = \sin(\num{1,25} \cdot{t_n})$ als gewichtete Impulsfolge in dem zuvor definierten Intervall $L$. \\
	\addimg{./assets/Lab2_53b.png}
	
	\item Die Anzahl der gewichteten Impulsen pro Intervall $L$ soll erhöht werden, dazu können mit dem Matlab-Befehl \inlcode{interp1}{} Zwischenwerte interpoliert\footnote{Zu gegebenen diskreten Daten (z. B. Messwerten) soll eine stetige Funktion (die sogenannte Interpolante oder Interpolierende) gefunden werden, die diese Daten abbildet.} werden. Unter der Benutzung der Optionen $original$, $nearest$, $linear$, $pchip$, $spline$ für \inlcode{interp1}{}, soll diese mit dem Matlab-Befehl \inlcode{resample}{} verglichen werden. \\
\end{enumerate}