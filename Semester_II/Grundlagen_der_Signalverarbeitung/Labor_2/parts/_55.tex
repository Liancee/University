Gegeben ist die Tondatei \textbf{ring.wav}.

\begin{enumerate}[label=\alph*)]
	\lstset{language=matlab}
	\item Einlesen der Tondatei und anschließendes Darstellen des Amplitudenvektors als Signal. \\
	Zum Einlesen der Audio Datei wurde die Funktion \inlcode{audioread}{filename, samples} benutzt. Die Parameter für den Matlab-Befehl sind der Pfad zur Datei und die Abtastfrequenz. Als Ergebnis gibt der Matlab-Befehl die abgetasteten Werte und die zugehörige Abtastfrequenz $f_s$ zurück. \vspace{4pt}\\
	\addimg{./assets/Lab2_55a.png}
	
	\item Bestimmung der Abtastfrequenz. \\
	Wie in der vorherigen Aufgabe bereits angesprochen, gibt der Matlab-Befehl \inlcode{audioread}{filename, samples} die Abtastfrequenz $f_s$ als Ergebnis zurück. Für die gegebene Tondatei beträgt die Abtastfrequenz
	\begin{equation*}
		f_s = \SI{11025}{\Hz}
	\end{equation*}
	
	\item Umwandlung des Signals mit höherer Abtastfrequenz \\
	Das Signal soll nun mit der Abtastfrequenz $f_{s,neu} = 2 \cdot{f_s}$ interpoliert werden. Umgesetzt wurde dies mit dem Matlab-Befehl \inlcode{resample}{y, 2, 1}. Dabei ist y der bereits bekannte Amplitudenvektor, 2 der Faktor für  die neue Abtastfrequenz und das letzte Parameter muss 1 sein, da wir die Abtastfrequenz verdoppeln (upsamplen, siehe \ref{resample}). \vspace{4pt}\\
	\addimg{./assets/Lab2_55b.png}
\end{enumerate}

\textbf{Matlab Code:}
\lstinputlisting[language=Matlab]{./assets/Lab2_55.m}