Given is the discrete signal:
\begin{equation*}
	s[n] = \cos[\Omega_0\cdot(n+P_0)+\theta_0]
\end{equation*}

\begin{enumerate}
	\item Calculation of the period duration $N_0$ for the following cases:
	
	\begin{nscenter}
		\begin{tabular}{c|lll}
			& $\Omega_0$ & $P_0$ & $\theta_0$ \\
			\hline
			1 & $\pi/{3}$ & $0$ & $2\pi$ \\
			2 & $3\pi/4$ & $2$ & $\pi/4$ \\
			3 & $3/4$ & $1$ & $1/4$ 
		\end{tabular}
	\end{nscenter}
	
	The normalization rule $\Omega_0$ is used to normalize a fundamental frequency $f_0$ to the sampling frequency $f_s$.
	\begin{equation*}
		\Omega_0 = 2\pi\frac{f_0}{f_s}
	\end{equation*}
	Following the definition of the circular frequency for continuous signals, the quotient of the frequencies in the normalization rule can also be expressed with the period duration for discrete signals
	\begin{equation*}
		\omega_0 = \frac{2\pi}{T} \rightarrow \Omega_0 = \frac{2\pi}{N_0}
	\end{equation*}
	converted to $N_0$
	\begin{equation*}
		N_0 = \frac{2\pi}{\Omega_0}
	\end{equation*}
	Calculations:
	{
		\setlength{\abovedisplayskip}{4pt}
		\setlength{\belowdisplayskip}{6pt}
		\setlength{\abovedisplayshortskip}{0pt}
		\setlength{\belowdisplayshortskip}{0pt}
		\begin{flalign*}
			&N_{0_{1}} = \frac{2\pi}{\frac{\pi}{3}} = \frac{2\pi\cdot3}{\pi} = 6 &
			&N_{0_{2}} = \frac{2\pi}{\frac{3\pi}{4}} = \frac{2\pi\cdot4}{3\pi} = \frac{8}{3} &
			&N_{0_{3}} = \frac{2\pi}{\frac{3}{4}} =   \frac{2\pi\cdot4}{3} = \frac{8\pi}{3} & 
		\end{flalign*}
	}
	
	\item Given are the discrete signals:
	\begin{align*}
		s_x[n] = \cos[\Omega_x\cdot(n+P_x)+\theta_x] \\
		s_y[n] = \cos[\Omega_y\cdot(n+P_y)+\theta_y] \\
	\end{align*}
	and the parameters	
	\begin{nscenter}
		\begin{tabular}{c|lll|lll}
			& $\Omega_x$ & $P_x$ & $\theta_x$ 
			& $\Omega_y$ & $P_y$ & $\theta_y$ \\
			\hline
			1 & $\pi/{3}$ & $0$ & $2\pi$ 
			& $8\pi/{3}$ & $0$ & $0$ \\
			2 & $3\pi/4$ & $2$ & $\pi/4$
			& $3\pi/4$ & $1$ & $-\pi$ \\
			3 & $3/4$ & $1$ & $1/4$
			& $3/4$ & 0 & 1
		\end{tabular}
	\end{nscenter}
	
	We get the functions:
	{
		\setlength{\abovedisplayskip}{0pt}
		\setlength{\belowdisplayskip}{6pt}
		\setlength{\abovedisplayshortskip}{0pt}
		\setlength{\belowdisplayshortskip}{0pt}
		\begin{align*}
			s_{x_1}[n] = \cos\left[\frac{\pi}{3}\cdot\left(n-0\right)+2\pi\right] & \qquad 
			s_{y_1}[n] = \cos\left[\frac{8\pi}{3}\cdot\left(n-0\right)+0\right] \\
			%
			s_{x_2}[n] = \cos\left[\frac{3\pi}{4}\cdot\left(n-2\right)+\frac{\pi}{4}\right] & \qquad 
			s_{y_2}[n] = \cos\left[\frac{3\pi}{4}\cdot\left(n-1\right)-\pi\right] \\
			%
			s_{x_3}[n] = \cos\left[\frac{3}{4}\cdot\left(n-1\right)+\frac{1}{4}\right] & \qquad 
			s_{y_3}[n] = \cos\left[\frac{3}{4}\cdot\left(n-0\right)+1\right]
		\end{align*}
	}
	\subsubsection{Solution}
	The aim is to check for which of the given combination(s) it is true that $s_{x_i}[n]$ is identical to $s_{y_i}[n]$.
	
	{
		\setlength{\abovedisplayskip}{0pt}
		\setlength{\belowdisplayskip}{6pt}
		\setlength{\abovedisplayshortskip}{0pt}
		\setlength{\belowdisplayshortskip}{0pt}
		
		\underline{Signal combination 1:}
		
		\begin{minipage}[t]{0.499999\linewidth}
			\begin{flalign*}
				s_{x_1}[n] & = \cos\left[\frac{\pi}{3}\cdot\left(n-0\right)+2\pi\right] & \\
				&=\cos\left[\frac{\pi}{3}n+2\pi\right] & \\
				&=\cos\left[\frac{\pi}{3}n\right] &
			\end{flalign*}
		\end{minipage}
		\begin{minipage}[t]{0.499999\linewidth}
			\begin{flalign*}
				s_{y_1}[n] & = \cos\left[\frac{8\pi}{3}\cdot\left(n-0\right)+0\right] & \\
				&= \cos\left[\frac{8\pi}{3}n\right] &
			\end{flalign*}
		\end{minipage}
		\vspace{2pt} \\
		$\Rightarrow$ The signal combination has only the same amplitude, but differs in frequency.
		
		\vspace{4pt}
		\underline{Signal combination 2:}
		
		\begin{minipage}[t]{0.499999\linewidth}
			\begin{flalign*}
				s_{x_2}[n] &= \cos\left[\frac{3\pi}{4}\cdot\left(n-2\right)+\frac{\pi}{4}\right] & \\
				&=\cos\left[\frac{3\pi}{4}n - \frac{3\pi}{2} + \frac{\pi}{4}\right] & \\
				&=\cos\left[\frac{3\pi}{4}n - \frac{5\pi}{4}\right] &
			\end{flalign*}
		\end{minipage}
		\begin{minipage}[t]{0.499999\linewidth}
			\begin{flalign*}
				s_{y_2}[n] &= \cos\left[\frac{3\pi}{4}\cdot\left(n-1\right)-\pi\right] & \\
				&=\cos\left[\frac{3\pi}{4}n - \frac{3\pi}{4} - \pi\right] & \\
				&=\cos\left[\frac{3\pi}{4}n - \frac{7\pi}{4}\right] &
			\end{flalign*}
		\end{minipage}
		\vspace{2pt} \\
		$\Rightarrow$ The signal combination has the same frequency and amplitude, but differs by a phase shift of $\SI{90}{\degree}$, $\pi/2$.
		
		\vspace{4pt}
		\underline{Signal combination 3:}
		
		\begin{minipage}[t]{0.499999\linewidth}
			\begin{flalign*}
				s_{x_3}[n] &= \cos\left[\frac{3}{4}\cdot\left(n-1\right)+\frac{1}{4}\right] & \\
				&=\cos\left[\frac{3}{4}n - \frac{3}{4} + \frac{1}{4}\right] & \\
				&=\cos\left[\frac{3}{4}n - \frac{1}{2}\right] &
			\end{flalign*}
		\end{minipage}
		\begin{minipage}[t]{0.5\linewidth}
			\begin{flalign*}
				s_{y_3}[n] &= \cos\left[\frac{3}{4}\cdot\left(n-0\right)+1\right] & \\
				&=\cos\left[\frac{3}{4}n - 0 + 1\right] & \\
				&=\cos\left[\frac{3}{4}n + 1\right] &
			\end{flalign*}
		\end{minipage}
		\vspace{2pt} \\
		$\Rightarrow$ The signal combination has the same frequency and amplitude, but has a phase shift of $\SI{0,5}{\radian}$.
	}	
\end{enumerate}
\clearpage