The signals $s_1(t)$ and $s_2(t)$ are given. It is to be checked whether these signals are even or odd.
\pgfplotsset{
	axis413/.style={
		clip=false,
		axis lines=middle,
		xmin=-6.5, xmax=6.5,
		xlabel={$t$},
		x label style={at={(current axis.right of origin)},	anchor=east, right=1mm},
		xtick distance={1},
		ymin=0, ymax=1.3,
		y=2.5cm,
		ylabel={$s(t)$},
		y label style={at={(current axis.above origin)}, anchor=south },
		yticklabels={$\hat{s}$ = -1, $\hat{s}$ = 1},
		ytick={-1,1},
		hide obscured x ticks=false,
	},
}
\begin{figure}[H]
	\centering
	\begin{tikzpicture}
		\begin{groupplot}[			
			group style={
				group size=1 by 2,
				xlabels at=edge bottom,
				ylabels at=edge left,
				vertical sep=+48pt, group name=plots,
			},
			width=0.8\linewidth,
		]
			\pgfplotsset{
				every axis plot/.append style={line width=1pt, mark=none, samples=10}
			}
			
			\nextgroupplot[axis413, ylabel={$s_1(t)$}]
			\addplot[color=blue] coordinates {(-6,0) (-3,0) (0,1) (3,0) (6,0)};
			
			\nextgroupplot[axis413, ylabel={$s_2(t)$}, ymin=-1.3]
			\addplot[color=blue] coordinates {(-6,0) (-3,0) (-3,-1) (3,1) (3,0) (6,0)};
		\end{groupplot}
	\end{tikzpicture}
	\caption{\label{413}$s_1(t)$ and $s_2(t)$}
\end{figure}

$\Rightarrow s_1(t)$ is an even function because it is symmetrical to the y-axis, or also because two x-values can be assigned to each y-value.
\begin{equation*}
	f(x) = f(-x)
\end{equation*}

$\Rightarrow s_2(t)$ is an odd function, as it is point-symmetrical to the origin of the coordinates, or each y-value can be assigned an x-value.
\begin{equation*}
	f(x) = -f(-x)
\end{equation*}
\clearpage